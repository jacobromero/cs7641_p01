\documentclass[
	%a4paper, % Use A4 paper size
	letterpaper, % Use US letter paper size
]{IEEEtran}


\author{Jacob Romero}
\title{Project 1: Supervised learning}

\begin{document}
	%\lsstyle
	\maketitle
	
	\begin{abstract}
		Machine learning has become more pervasive in our everyday life, used for everything from personalized advertising to making financial decisions. In this paper I will compare multiple machine learning algorithms against two datasets, both related to predicting financial information from individuals data. These comparisons will look at how each of the machines learning algorithms performs as more training data is introduced, how each algorithms performs as a function of training time, and lastly a comparison of the algorithms against each other.
	\end{abstract}
	
	\section{Introduction and Problems}
	In this paper we will look at two datasets, and compare these data sets to five different machine learning algorithms. These five algorithms are decision trees, neural networks, boosted decision trees, support vector machines, and k-nearest neighbors. Each algorithm has different properties in the sense of how it models a problem space, its preference bias, expression bias, how fast training and prediction can be done.
	
	The algorithms described previously will be ran and tested against two datasets. The first data set is data from the U.S. census regarding various attributes of an individual. These attributes include areas such as age, education level, marital status, occupation, race, sex, etc. From these attributes the final target is whether the individual makes less than or equal to 50 thousand dollars a year, or more than this amount. With a model that can accurately predict an individuals income, or income class in our case would have many uses in other areas, such as targeted marketing, personalized ads, or credit decisions.
	
	Continuing along the same thing of using machine learning for financial motivations, the second dataset uses a similar 
\end{document}